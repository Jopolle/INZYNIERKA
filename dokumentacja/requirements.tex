\chapter{Wymagania projektu}
\label{chap:wymagania-projektu}

Przed opisaniem szczegółowych informacji dotyczących sprzętu i oprogramowania, należałoby zdefiniować wymagania dla projektowanego stanowiska. Wymagania mają tutaj dwa
poziomy: funkcjonalny (co stanowisko ma potrafić) oraz niefunkcjonalny (z jaką dokładnością, w jaki
sposób i przy jakiej wygodzie użytkowania). Poniższy zestaw założeń był punktem odniesienia przy
projektowaniu zarówno płytki PCB, jak i firmware’u opisanych w kolejnych rozdziałach.

\subsection{Wymagania funkcjonalne}

Podstawowym celem jest budowa stanowiska, które pozwala na wygodny pomiar i generację
sygnałów \mbox{0--10~V} typowych dla instalacji HVAC oraz na rozszerznie działania prostego układu
regulacji temperatury. W praktyce sprowadza się to do spełnienia następujących wymagań:

\begin{itemize} 
  \item Płytka powinna umożliwiać pomiar napięć wejściowych w zakresie \mbox{0--10~V}
        z efektywną rozdzielczością nie gorszą niż 10~mV na kanał,
        tak aby możliwe było rejestrowanie zmian wielkości fizycznych z dokładnością
        wystarczającą do ćwiczeń laboratoryjnych.
  \item Układ ma generować na każdym z ośmiu wyjść sygnał \mbox{0--10~V} o obciążalności
        co najmniej 5~mA, co pozwala na sterowanie typowymi przetwornikami i siłownikami
        \mbox{0--10~V} spotykanymi w systemach HVAC (zawory, przepustnice, falowniki itp.).
  \item Do dyspozycji użytkownika powinno być co najmniej osiem wejść i osiem wyjść
        analogowych, tak aby dało się równolegle podłączyć kilka czujników i elementów
        wykonawczych oraz odtworzyć uproszczony schemat ukłądu sterowania małej centrali wentylacyjnej.
  \item Każdy kanał wejściowy ma udostępniać, oprócz samego sygnału, także dedykowaną masę
        sygnałową oraz zasilanie \mbox{+24~V}, co umożliwi bezpośrednie podłączenie aktywnych
        czujników pomiarowych bez dodatkowych zasilaczy pomocniczych.
  \item Na wyświetlaczu TFT powinny być pokazywane aktualne wartości napięć na wejściach
        i wyjściach \mbox{0--10~V}, wraz z ich opisem funkcjonalnym (np. ,,T supply'',
        ,,T extract'', ,,Heater'', ,,Cooler''), tak aby z poziomu GUI dało się zorientować,
        jaką rolę pełni dany kanał w danej konfiguracji.
  \item Interfejs użytkownika ma umożliwiać ustawienie zadanej temperatury oraz granic pasm sekwencji (grzanie, chłodzenie, odzysk
        ciepła, pasmo martwe), a następnie obserwację wynikowych sygnałów wyjściowych
        \mbox{0--10~V}.
  \item Oprogramowanie powinno realizować co najmniej jedną pętlę regulacji temperatury (regulator PI/PID w zależności od wybranej konfiguracji) z wyjściem wyrażonym w procentach, które są następnie rozdzielane
        na wyjścia odpowiadające nagrzewnicy, chłodnicy, odzyskowi ciepła i wentylatorowi
        zgodnie z wybraną sekwencją pracy.
  \item Użytkownik powinien mieć możliwość przełączania się pomiędzy co najmniej dwiema
        kompletnymi konfiguracjami stanowiska (różne nastawy, parametry regulatora,
        mapowanie kanałów, progi sekwencji) bez konieczności rekompilacji oprogramowania
        -- konfiguracja ma być wczytywana z opisu tekstowego (JSON) w czasie działania
        systemu.
\end{itemize}

Realizacja powyższych punktów oznacza, że stanowisko może zostać wykorzystane jako prosty,
ale dość elastyczny w dalszym rozwoju testowy zestaw labolatoryjny, pozwalające zarówno na pomiary statyczne, jak i na
podstawowe eksperymenty z regulacją temperatury i sekwencjami działania układu HVAC.

\subsection{Wymagania niefunkcjonalne}

Drugą grupę stanowią wymagania niefunkcjonalne, związane z dokładnością, bezpieczeństwem,
czytelnością dydaktyczną oraz możliwością dalszego rozwijania projektu. W pracy przyjęto
w szczególności następujące założenia:

\begin{itemize}
  \item Tor pomiarowy \mbox{0--10~V} powinien zapewniać rozdzielczość co najmniej 10~mV oraz
        błąd podstawowy rzędu pojedynczych dziesiątek miliwoltów w całym zakresie, tak aby
        wyniki pomiarów były powtarzalne i nadawały się do prostych zadań obliczeniowych
        (np. wyznaczanie charakterystyk statycznych czujników).
  \item Układ ma być odporny na typowe błędy popełniane w laboratorium: odwrotną polaryzację
        zasilania, przypadkowe zwarcia oraz krótkotrwałe przekroczenia dopuszczalnego
        napięcia na wejściach/wyjściach. Wymusza to zastosowanie zabezpieczeń w torze
        zasilania, ochrony przeciwprzepięciowej oraz podstawowych środków ochrony ESD.
  \item Projekt PCB powinien uwzględniać dobre praktyki dla układów mieszanych: wydzielenie
        stref analogowej i cyfrowej, kontrolowany sposób łączenia mas, sensowne prowadzenie
        ścieżek zasilania i linii szybkich (SPI). Dzięki temu stanowisko ma zachowywać się
        stabilnie także przy jednoczesnej pracy kilku kanałów i podczas dynamicznych zmian
        sygnałów.
  \item Zarówno schemat, jak i sam układ na płytce powinny być czytelne dla studenta. Przepływ
        sygnału (wejścia \mbox{0--10~V} -- przetwornik A/C -- mikrokontroler -- przetwornik C/A --
        wyjścia \mbox{0--10~V}) powinien dać się łatwo prześledzić wzrokowo, a opisy na
        warstwie nadruku i w GUI mają pomagać w zrozumieniu roli poszczególnych bloków.
  \item Oprogramowanie powinno być oparte na otwartych, powszechnie stosowanych narzędziach
        (w tym przypadku: Zephyr RTOS, LVGL) i napisane w sposób umożliwiający dalszy rozwój: dołożenie
        kolejnych ekranów HMI, nowych algorytmów regulacji lub innego sposobu wczytywania
        konfiguracji (np. z karty SD) bez konieczności radykalnej przebudowy kodu.
  \item Projekt nie zakłada pełnej certyfikacji pod kątem EMC, ale sposób prowadzenia mas,
        filtracja zasilania i zastosowane zabezpieczenia mają ograniczać podatność na
        zakłócenia do poziomu akceptowalnego w warunkach laboratorium dydaktycznego.
\end{itemize}

Tak zdefiniowany zestaw wymagań wyznacza ramy dalszej części pracy. Rozdział dotyczący
projektu części sprzętowej pokazuje, w jaki sposób spełniono założenia związane z torami
\mbox{0--10~V}, zasilaniem i PCB, natomiast rozdział o firmware opisuje realizację wymagań dotyczących
regulacji, interfejsu użytkownika oraz mechanizmu wczytywania konfiguracji.
