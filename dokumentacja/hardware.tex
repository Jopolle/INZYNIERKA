\chapter{Projekt części sprzętowej}
\label{sec:wymagania-sprzetowe-przeglad}

\section{Wymagania sprzętowe — przegląd}

Projektowana płytka stanowi jedną, spójną platformę do pomiaru i generacji sygnałów 0--10~V w aplikacjach HVAC\cite{Analog_0_10V_Guideline}, współpracującą z zestawem uruchomieniowym STM32F746G-DISCO wyposażonym m.in.\ w mikrokontroler STM32F7 i panel TFT\cite{ST_STM32F746NG_Datasheet,ST_32F746GDISCOVERY}. Rozwiązanie to ma pełnić rolę uniwersalnego ,,front-endu'' analogowego dla laboratoryjnego sterownika HVAC: umożliwia zarówno rejestrację sygnałów z czujników i przetworników 0--10~V, jak i generację ośmiu niezależnych sygnałów 0--10~V do sterowania elementami wykonawczymi (siłowniki przepustnic, zawory mieszające, przetwornice wentylatorów itp.).

Od strony zasilania przewidziano instalacyjne wejście DC (nominalnie 24~V) z zabezpieczeniem (odwrotna polaryzacja, przepięcia, wstępna filtracja), a następnie podział zasilania na osobne gałęzie dla części cyfrowej, analogowej i elementów interfejsowych. Taki układ zmniejsza wpływ zakłóceń na pomiary i stabilizuje pracę wyjść analogowych. Interfejs pomiarowy obejmuje osiem wejść 0--10~V przygotowanych do bezpiecznego próbkowania przez przetwornik A/C oraz osiem wyjść 0--10~V realizowanych przez przetwornik C/A i wzmacniacze operacyjne.

Istotnym założeniem projektowym było zachowanie kompatybilności elektrycznej i mechanicznej z płytką STM32F746G-DISCO. W centralnej części PCB przewidziano złącze typu goldpin (standard Dupont), którego raster i położenie odpowiada złączu rozszerzeń zestawu uruchomieniowego. Dzięki temu całość tworzy układ piętrowy: płytka z analogowym interfejsem pełni funkcję karty pomiarowej, a zestaw uruchomieniowy zapewnia moc obliczeniową oraz interfejs użytkownika (TFT z panelem dotykowym).

\begin{figure}[htbp]
  \centering
  \includesvg[inkscapelatex=false,width=\textwidth]{img/pcb/pcb_drawing.svg}
  \caption{Widok płytki PCB z zaprojektowanym rozmieszczeniem elementów i złącz. W centralnej części znajduje się obszar montażu płytki STM32F746G-DISCO.}
  \label{fig:pcb-obraz}
\end{figure}

\begin{figure}
  \centering
  \includegraphics[width=0.75\textwidth]{img/render_full.png}
  \caption{Schemat blokowy głównych modułów funkcjonalnych płytki z interfejsem 0--10~V.}
  \label{fig:block-diagram}
\end{figure}

\section{Moduł zasilania}

Układ zasilania płytki został zaprojektowany tak, aby bezpiecznie przyjąć instalacyjne napięcie stałe (do ok.\ 24~V) i rozdzielić je na dwa napięcia pomocnicze: +5~V oraz +3{,}3~V. Schemat modułu zasilania przedstawiono na rysunku~\ref{fig:schemat-zasilanie}.

Na wejściu zastosowano gniazdo J1 (DC~jack 5{,}5$\times$2{,}1~mm), do którego doprowadzane jest napięcie z zewnętrznego zasilacza. Bezpośrednio za złączem znajduje się polimerowy bezpiecznik samoresetujący F1 (PPTC~1{,}1~A/30~V), pełniący rolę zabezpieczenia nadprądowego w przypadku zwarcia na płytce lub błędnego podłączenia odbiorników. Równolegle do wejścia umieszczono diodę TVS D3 (SMBJ33A) tłumiącą przepięcia.

Ochronę przed odwrotną polaryzacją zasilania zrealizowano w oparciu o tranzystor P-MOSFET mocy Q1 (AO4407A w obudowie SO-8 lub równoważny)\cite{AO4407A}. Tranzystor włączono w konfiguracji ,,idealnej diody'': jego źródło jest połączone z wejściem zasilania, dren z resztą układu, a bramka sterowana jest poprzez rezystor R4 i diodę Zenera D4 (1N4742A)\cite{Goodark_1N47xxA}. Przy poprawnej polaryzacji tranzystor przewodzi z minimalnym spadkiem napięcia na kanale, a w przypadku odwrotnego podłączenia zasilacza blokuje przepływ prądu i chroni dalsze stopnie zasilania.

Za sekcją ochronną pracuje wyłącznik SW1 odcinający cały moduł zasilania. Dioda LED D5 z rezystorem szeregowym R14 sygnalizuje obecność napięcia po stronie wejściowej; pozwala to na szybką kontrolę stanu zasilania przed przetwornicami.

Kolejnym etapem jest wstępna filtracja C-L-C napięcia wejściowego. Dławik L3 (33~\textmu H) oraz kondensatory C15 (100~\textmu F/63~V), C26 (100~nF), C27 (47~\textmu F) i C16 (1~\textmu F) tworzą filtr typu PI, ograniczający wahania napięcia oraz szpilki napięciowe związane z pracą przetwornic impulsowych. Dodatkowy koralik ferrytowy FB1, włączony szeregowo w linii 24~V, poprawia tłumienie zakłóceń o wyższych częstotliwościach, które mogłyby przenikać do dalszych części instalacji.

Konwersję napięcia na poziomy logiczne realizują dwie niezależne przetwornice buck z rodziny LM2596S\cite{TI_LM2596}. Układ U1 (LM2596S-5) generuje linię +5~V. W jego torze znajdują się dławik L2 (33~\textmu H), dioda Schottky'ego D1 (SK56) oraz kondensatory wejściowe C2 (100~\textmu F/63~V) i C1 (470~nF) oraz wyjściowe C4 i C3. Analogicznie układ U2 (LM2596S-3{,}3) dostarcza linię +3{,}3~V z użyciem dławika L1 (68~\textmu H), diody D2 (SK56) oraz zestawu kondensatorów C5, C6 po stronie wejściowej i C7, C8 po stronie wyjściowej. Elementy zostały dobrane zgodnie z zaleceniami producenta, tak aby zapewnić stabilność pętli regulacji dla zakładanych obciążeń oraz odpowiednio niski poziom tętnień.

Zastosowanie dwóch niezależnych przetwornic zamiast jednego źródła z liniowymi stabilizatorami wtórnymi ma kluczowe znaczenie w kontekście sprawności i wydzielania ciepła. Przy typowych prądach pobieranych przez mikrokontroler, przetwornik A/C i przetwornik C/A, bezpośrednia konwersja z 24~V na 5~V/3{,}3~V w stabilizatorach liniowych skutkowałaby znaczącym poborem mocy. Przetwornice impulsowe LM2596S pozwalają ograniczyć straty do pojedynczych watów nawet przy pełnym obciążeniu linii 5~V i 3{,}3~V.

Analogicznie do sygnalizacji obecności napięcia na linii 24~V, stan szyn wyjściowych jest sygnalizowany diodami LED: D7 (dla linii +5~V) z rezystorem R16 oraz D6 (dla linii +3{,}3~V) z rezystorem R15.

Dodatkowo, w bezpośrednim sąsiedztwie przetwornic zasilających przewidziano punkty testowe dla linii +5~V oraz +3{,}3~V. Każdy punkt testowy ma obok wyprowadzoną również masę, co umożliwia wygodny pomiar napięć roboczych sondą oscyloskopową lub multimetrem podczas uruchamiania i diagnostyki układu. Lokalizacja tych punktów przy samych przetwornicach pozwala na obserwację rzeczywistych tętnień i zachowania regulatorów bez dodatkowego wpływu rezystancji i indukcyjności ścieżek zasilających.

Całość tworzy spójny tor:
\emph{wejście i zabezpieczenia $\rightarrow$ filtracja wstępna $\rightarrow$ konwersja 24~V na +5~V/+3{,}3~V $\rightarrow$ dystrybucja i sygnalizacja},
co przekłada się na stabilną pracę układu.

Na poziomie PCB cały moduł zasilania został umieszczony w górnej części płytki (rysunek~\ref{fig:pcb-obraz}), możliwie blisko gniazda wejściowego oraz oddalony od wrażliwych części analogowych. Ścieżki prowadzące prądy impulsowe z przetwornic LM2596S zaprojektowano jako szerokie i możliwie krótkie, z lokalnymi polami masy minimalizującymi powierzchnię pętli prądowych. Dzięki temu ograniczono emisję zakłóceń przewodzonych i promieniowanych oraz uproszczono separację pomiędzy strefą mocy a strefą pomiarową.

Zastosowanie przetwornic impulsowych LM2596S zamiast stabilizatorów liniowych (np.\ serii 78xx\cite{ST_L78xx} lub LM317\cite{TI_LM317}) wynika bezpośrednio z warunków pracy projektowanej płytki. Przy typowym napięciu wejściowym 24~V oraz konieczności zasilenia mikrokontrolera, przetworników A/C i C/A oraz interfejsu użytkownika, stabilizatory liniowe musiałyby rozproszyć znaczną moc na elementach półprzewodnikowych, co prowadziłoby do dużego nagrzewania, wymagało dodatkowych radiatorów i groziło poparzeniu przy dłuższej pracy. Przetwornice obniżające napięcie LM2596S pozwalają uzyskać te same poziomy napięć 5~V i 3{,}3~V przy znacznie wyższej sprawności, co przekłada się na mniejsze straty mocy i niższą temperaturę pracy. Dodatkowo rodzina LM2596 jest dobrze udokumentowana i szeroko stosowana w aplikacjach laboratoryjnych, co ułatwia dobór elementów zewnętrznych oraz przewidywalne zachowanie układu przy różnych konfiguracjach obciążenia.

Z punktu widzenia generacji sygnału 0--10~V istotne są również pozostałe parametry rodziny TLV9304. Są to wzmacniacze niskoszumowe, o wyjściu typu rail--to--rail i szerokim zakresie napięcia zasilania (od ok.\ 4{,}5~V do 40~V)\cite{TI_TLV9304}. Przy zasilaniu z instalacyjnych 24~V pozwala to uzyskać użyteczny zakres wyjściowy bardzo blisko szyn zasilania, co zmniejsza błąd przy skrajnych poziomach napięcia 0~V i 10~V bez konieczności stosowania dodatkowych przetwornic. Niewielkie napięcie niezrównoważenia oraz mały dryft temperaturowy są w pełni wystarczające dla typowych aplikacji HVAC, gdzie ważniejsza od absolutnej precyzji jest stabilność i powtarzalność sygnału sterującego. Dodatkowo wzmacniacze te potrafią dostarczyć prąd wyjściowy rzędu kilkudziesięciu miliamperów, co upraszcza współpracę z typowymi wejściami 0--10~V przetwornic częstotliwości i sterowników automatyki budynkowej, obciążających wyjście stosunkowo niską rezystancją.

\begin{figure}[htbp]
  \centering
  \includesvg[inkscapelatex=false,width=\textwidth]{img/schemat_zasilanie.svg}
  \caption{Moduł zasilania płytki z wejściem instalacyjnym 24~V, torem ochrony przed przepięciami i odwrotną polaryzacją oraz przetwornicami step-down LM2596 generującymi linie +5~V i +3{,}3~V.}
  \label{fig:schemat-zasilanie}
\end{figure}

\section{Tor wyjściowy \texorpdfstring{0--10\,V}{0-10 V}: przetwornik DAC + wzmacniacze operacyjne}

Tor wyjściowy generujący sygnały 0--10~V oparto na ośmiokanałowym przetworniku cyfrowo-analogowym U3 (DAC7568IAPW)\cite{TI_DAC7568} współpracującym z dwoma czterokanałowymi wzmacniaczami operacyjnymi U5 i U6 (TLV9304xPW)\cite{TI_TLV9304}. Dzięki temu możliwe jest niezależne sterowanie wszystkimi ośmioma wyjściami analogowymi. Schemat toru wyjściowego przedstawiono na rysunku~\ref{fig:schemat-dac}.

\begin{figure}[htbp]
  \centering
  \includesvg[inkscapelatex=false,width=\textwidth]{img/schemat_dac.svg}
  \caption{Schemat toru wyjściowego 0--10~V wraz z wzmacniaczami operacyjnymi.}
  \label{fig:schemat-dac}
\end{figure}

Przetwornik DAC7568 jest zasilany z linii +3{,}3~V. Komunikację z płytką STM32F746G-DISCO realizuje poprzez magistralę SPI: linie \texttt{NSS\_OUT}, \texttt{MOSI\_OUT} i \texttt{SCK\_OUT} zostały wyprowadzone z MCU i doprowadzone do odpowiednich pinów układu U3. Z uwagi na charakter przetwornika (układ typu ,,write-only'') nie przewidziano linii MISO; konfiguracja rejestrów i aktualizacja wyjść odbywa się wyłącznie poprzez wysyłanie ramek danych z mikrokontrolera.

Wyprowadzenie \texttt{VREFIN/VREFOUT} służy do ustalenia napięcia referencyjnego. W projekcie wykorzystano wewnętrzne źródło odniesienia przetwornika, dlatego pin został odsprzęgnięty kondensatorem C17 (150~nF) umieszczonym możliwie blisko wyprowadzeń, zgodnie z zaleceniami producenta\cite{TI_DAC7568}. Masę części analogowej doprowadzono do masy analogowej GNDA, która na PCB prowadzona jest jako wydzielona wyspa z kontrolowanym połączeniem do wspólnej masy GND poprzez elementy R2/C11 (opisane szerzej w podrozdziale o torze wejściowym).

Każdy z ośmiu kanałów wyjściowych DAC (VOUTA--VOUTH) jest dalej kształtowany przez prosty filtr dolnoprzepustowy RC na wejściu wzmacniacza: rezystor szeregowy (R6--R13, typowo 3{,}3~k$\Omega$) oraz kondensator do masy (C18, C21--C25, 1~\textmu F). Wyznacza to częstotliwość odcięcia rzędu
\begin{equation}
f_c \approx \frac{1}{2\pi R C} \approx \frac{1}{2\pi \cdot 3{,}3\ \mathrm{k}\Omega \cdot 1\ \mu\mathrm{F}} \approx 48\ \mathrm{Hz},
\end{equation}
co skutecznie tłumi sugnały pochodzące z aktualizacji DAC, a jednocześnie jest w pełni wystarczające dla powolnych procesów w systemach HVAC.

Wzmacniacze U5A--U5D oraz U6A--U6D pracują w konfiguracji nieodwracającej i są zasilane z linii +24~V (wg producenta maksymalne napięcie zasilania to 40~V DC). W ramach zabezpieczenia na torze zasilającym idącym do wzmacniaczy znajduje się bezpiecznik polimerowy samoresetujący F2 oraz filtr dolnoprzepustowy C-R-C składający się z rezystora R17, kondensatorów elektrolitycznych C29 i C31 oraz kondensatorów ceramicznych C28 i C30. Filtr usuwa wszelkie niepożądane zakłócenia, które mogą występować w torze zasilania.

Zastosowanie wzmacniaczy o szerokim zakresie napięć zasilania pozwala uzyskać odpowiedni zapas napięciowy dla wyjść 0--10~V bez konieczności stosowania dodatkowych przetwornic podwyższających. Dla każdego kanału zastosowano identyczną sieć sprzężenia zwrotnego: rezystor do masy RG (10~k$\Omega$) oraz rezystor w pętli sprzężenia RF (30~k$\Omega$). Wzmocnienie napięciowe pojedynczego toru wynosi więc
\begin{equation}
A_v = 1 + \frac{R_F}{R_G} = 1 + \frac{30\ \mathrm{k}\Omega}{10\ \mathrm{k}\Omega} = 4.
\end{equation}
Przy referencji DAC rzędu 2{,}5~V umożliwia to uzyskanie pełnego zakresu 0--10~V na wyjściu wzmacniacza, z zapasem na niewielkie tolerancje i błędy kalibracji.

Wyjścia poszczególnych wzmacniaczy są wyprowadzone na uniwersalne złącza J21--J28. Każde złącze udostępnia linię sygnałową 0--10~V, odniesienie SG (signal ground) oraz pin SHIELD przeznaczony do ekranowania przewodów.

Na płytce PCB (rysunek~\ref{fig:pcb-obraz}) złącza wyjściowe zostały rozmieszczone wzdłuż prawej krawędzi w regularnym rastrze, co ułatwia prowadzenie przewodów i daje gwarancje ,,przepływu'' sygnałów przez sterownik (od lewej do prawej). Ścieżki sygnałowe pomiędzy wzmacniaczami a złączami są możliwie krótkie i prowadzone nad ciągłą płaszczyzną masy, co redukuje indukcyjność pętli i przesłuchy pomiędzy kanałami.

Podsumowując, tor wyjściowy ma strukturę:
\emph{STM32 $\rightarrow$ DAC7568 $\rightarrow$ filtr RC $\rightarrow$ wzmacniacz nieodwracający o wzmocnieniu 4 $\rightarrow$ złącze sygnałowe z ekranem i zasilaniem pola},
co zapewnia zarówno elastyczność sterowania, jak i zgodność z powszechnie stosowanym standardem 0--10~V.

\section{Tor wejściowy \texorpdfstring{0--10\,V}{0-10 V}: interfejs pomiarowy}

Tor wejściowy odpowiada za obsługę czujników pomiarowych (pasywnych i aktywnych) oraz bezpieczne doprowadzenie sygnałów 0--10~V do wielokanałowego przetwornika A/C ADS8688\cite{TI_ADS8688}. Schemat tej części układu przedstawiono na rysunku~\ref{fig:schemat-mcu}. Przetwornik ten integruje w sobie przełączany multiplekser wejściowy, programowalne zakresy napięciowe oraz wewnętrzny front-end zabezpieczający, co pozwala uprościć zewnętrzny tor analogowy.

\begin{figure}[htbp]
  \centering
  \includesvg[inkscapelatex=false,width=\textwidth]{img/schemat_mcu.svg}
  \caption{Schemat toru wejściowego 0--10~V wraz z przetwornikami ADC oraz połączenia do mikrokontrolera.}
  \label{fig:schemat-mcu}
\end{figure}

Po lewej stronie schematu rozmieszczono osiem identycznych złączy wejściowych (oznaczonych jako Signal\_connector, J2--J9). Na każdym złączu wyprowadzono cztery piny:
\begin{itemize}
    \item SIG — linia sygnałowa 0--10~V,
    \item SG — dedykowany powrót sygnałowy (signal ground) dla danego kanału,
    \item SHIELD — ekran przewodu, przeznaczony do podłączenia oplotu lub ekranu kabla,
    \item +24~V — zasilanie czujnika/konwertera z instalacji 24~V.
\end{itemize}
Takie ustandaryzowane złącza pozwalają na łatwą zamianę czujników pomiędzy kanałami oraz ograniczają liczbę pomyłek przy okablowaniu.

Linia SIG każdego złącza jest prowadzona bezpośrednio do odpowiedniego pinu wejściowego przetwornika (AIN\_xP), natomiast powrót SG trafia do przypisanego ujemnego wejścia odniesienia kanału (AIN\_xGND). W ten sposób tworzony jest układ wejściowy o charakterze pseudo-różnicowym, w którym każdy kanał ma swój lokalny powrót odniesiony do tej samej masy analogowej GNDA. Taka topologia kompensuje spadki napięć i przesłuchy na przewodzie powrotnym oraz poprawia odporność na zakłócenia wspólne przy dłuższych odcinkach okablowania, co jest kluczowe w instalacjach HVAC rozproszonych na dużej przestrzeni.

W torze wejściowym nie stosuje się dodatkowych dzielników ani zewnętrznych filtrów antyaliasingowych; skalowanie i zabezpieczenie wejść realizuje wewnętrzny front-end przetwornika ADS8688\cite{TI_ADS8688}, który oferuje kilka przełączanych zakresów pomiarowych oraz obwody ograniczające prądy przy przepięciach.

Piny SHIELD wszystkich wejść są połączone do wspólnego ekranu wylanego pod ścieżkami sygnału i dołączonego względem masy przez filtr RC: rezystor R2 (1~M$\Omega$) równolegle z kondensatorem C11 (10~nF) pomiędzy GNDA i GND. Zapewnia to upływ ładunków statycznych oraz tłumienie składowych o wysokiej częstotliwości, a jednocześnie ogranicza stałoprądowe prądy pętli masy i chroni ekran przed ,,przeciąganiem'' potencjału przez inne urządzenia podłączone do tej samej instalacji. Rozwiązanie to stanowi jednocześnie kontrolowany punkt połączenia masy analogowej i cyfrowej — GNDA jest używana w torach wejściowych i wyjściowych, natomiast GND stanowi referencję dla logiki cyfrowej i zasilania 3{,}3~V.

Część analogowa przetwornika ADS8688\cite{TI_ADS8688} jest zasilana z linii +5~V (kondensator C9 1~\textmu F blisko pinów AVDD/AGND), natomiast część cyfrowa z linii +3{,}3~V (kondensator C10 10~\textmu F przy DVDD/DGND). Linia \texttt{RST/PD} jest podciągnięta rezystorem 10~k$\Omega$ do +3{,}3~V i może być sterowana z mikrokontrolera, co umożliwia programowe resetowanie przetwornika oraz przechodzenie w tryb uśpienia w stanie bezczynności.

Odniesienie napięciowe przetwornika realizowane jest wewnętrznie; piny REFCAP, REFIO i REFGND są odsprzęgnięte kondensatorami klasy X7R (C12 i C13 po 1~\textmu F) umieszczonymi możliwie blisko wyprowadzeń, zgodnie z zaleceniami producenta\cite{TI_ADS8688}. Zapewnia to niskoszumowe, stabilne napięcie odniesienia, co bezpośrednio przekłada się na rozdzielczość efektywną przetwornika.

Na schemacie przewidziano również dodatkowe złącze J18 w postaci listwy goldpin z możliwością założenia zworek. Złącze to służy do konfigurowania połączenia pomiędzy masą sygnałową (SG), wykorzystywaną jako powrót dla wejść 0--10~V, a ogólną masą układu (GND). Poprzez odpowiednie ustawienie zworek użytkownik może zdecydować, czy masa sygnałowa ma pozostać możliwie bez szumu i odniesiona głównie do GNDA (praca z czujnikami pasywnymi), czy też powinna zostać zwarta z masą zasilania w celu zasilania czujników aktywnych, wymagających wspólnego potencjału odniesienia dla toru zasilania i sygnału.

Takie rozwiązanie ma kilka zalet:
\begin{itemize}
    \item umożliwia elastyczną konfigurację toru wejściowego w zależności od typu podłączonych czujników (pasywne z odseparowaną masą sygnałową vs.\ aktywne wymagające wspólnego potencjału zasilania i sygnału),
    \item poprawia kompatybilność z typowymi przetwornikami 0--10~V, które zakładają wspólną masę zasilania i wyjścia napięciowego,
    \item pozwala na łatwe eksperymentowanie i diagnostykę w warunkach laboratoryjnych — zmiana konfiguracji sprowadza się do przełożenia zworek, bez konieczności modyfikacji PCB ani rozcinania ścieżek.
\end{itemize}

Komunikacja z mikrokontrolerem odbywa się po magistrali SPI: linie SDI, SDO, SCLK i CS zostały połączone z odpowiednimi pinami MCU. Zestaw STM32F746G-DISCO wyposażony jest w dwa oddzielne interfejsy SPI. Dzięki temu oba przetworniki (ADS8688 i DAC7568) mogą komunikować się z MCU niezależnie od siebie; każdy z nich korzysta z własnej magistrali, co upraszcza konfigurację sprzętową.

Na poziomie PCB (rysunek~\ref{fig:pcb-obraz}) przetwornik ADS8688 oraz złącza wejściowe zostały umieszczone w lewej części płytki, w niewielkiej odległości od siebie. Ścieżki sygnałowe SIG/SG prowadzone są nad pełną płaszczyzną masy GNDA, z zachowaniem odstępów pomiędzy kanałami, co ogranicza pojemnościowe sprzężenia krzyżowe. Ekrany SHIELD tworzą osobną, częściowo wylaną strefę połączoną z GNDA poprzez elementy R2/C11.

\section{Projekt PCB i separacja stref}

Widok płytki PCB przedstawiono na rysunku wygenerowanym z narzędzia CAD (rysunek~\ref{fig:pcb-obraz}). Płytka została zaprojektowana jako dwuwarstwowa, z rozległymi polami masy w obu warstwach. Główna powierzchnia zajmowana jest przez prostokątny obrys odpowiadający formatowi płytki STM32F746G-DISCO, w którego obrębie przewidziano wycięty obszar (strefę bez elementów i ścieżek) stanowiący miejsce montażu zestawu uruchomieniowego. W czterech narożach oraz w pobliżu złączy umieszczono otwory montażowe umożliwiające sztywne zamocowanie całości do płyty bazowej lub obudowy.

Pod względem funkcjonalnym płytkę można podzielić na trzy główne strefy:
\begin{enumerate}
    \item \textbf{Strefa zasilania} — w górnej części, obejmująca gniazdo zasilania J1, układ ochrony (F1, D3, D4, Q1, L3, FB1) oraz przetwornice LM2596S. Ścieżki o dużych prądach i impulsowych zmianach prądu prowadzone są lokalnie, nad fragmentarycznymi polami masy, co ogranicza emisję zakłóceń.
    \item \textbf{Strefa analogowa} — przy lewej i prawej krawędzi, obejmująca przetwornik ADS8688, przetwornik DAC7568, wzmacniacze TLV9304 oraz złącza wejściowe i wyjściowe. W tych częściach płytki zastosowano wydzieloną masę GNDA, która jest łączona z ogólną masą GND w jednym punkcie poprzez elementy R2/C11.
    \item \textbf{Strefa cyfrowa} — w centralnej części, wokół rastru GPIO mikrokontrolera, gdzie prowadzone są linie SPI. Tutaj masa GND jest wylana jako osobna wyspa, a przejścia do masy analogowej są kontrolowane.
\end{enumerate}

Istotnym elementem projektu PCB jest częste uwspólnianie potencjałów masy pomiędzy warstwami i strefami. W torach zasilania oraz w części cyfrowej zastosowano gęsto rozmieszczone przelotki łączące pola GND na obu warstwach, co skraca ścieżkę powrotu prądu i zmniejsza indukcyjność pętli. Analogicznie, w obszarach torów analogowych rozmieszczono przelotki spinające pola GNDA, dzięki czemu masa analogowa tworzy zwartą, dobrze przewodzącą referencję dla sygnałów pomiarowych i wyjściowych. Połączenie GNDA z GND pozostaje jednak zrealizowane w jednym, kontrolowanym punkcie (R2/C11), co łączy zalety wspólnego potencjału odniesienia z ograniczeniem pętli mas i przesłuchów pomiędzy strefą analogową a cyfrową.

Rozdzielenie tych stref w przestrzeni płytki ogranicza przesłuchy pomiędzy torami, a jednocześnie pozwala na intuicyjną analizę układu podczas uruchamiania i diagnostyki. Szerokości ścieżek dobrano zgodnie z przewidywanymi prądami (najszersze dla linii 24~V, 5~V i 3{,}3~V, węższe dla linii sygnałowych). Przy przejściach pomiędzy warstwami stosowane są przelotki w ,,gęstych'' grupach, tak aby zapewnić zwarty powrót prądu między płaszczyznami masy.

Dodatkowym elementem ułatwiającym pracę z płytką jest rozbudowana warstwa opisowa (silkscreen), na której zaznaczono nazwy złączy, kierunki numeracji pinów, oznaczenia kanałów wejściowych i wyjściowych oraz podstawowe kierunki przepływu sygnału (strzałki). Dzięki temu użytkownik może korzystać z płytki w warunkach laboratoryjnych praktycznie bez konieczności sięgania do schematu.

Opisany w niniejszym rozdziale projekt części sprzętowej stanowi bazę dla dalszych rozważań dotyczących implementacji algorytmów sterowania i architektury oprogramowania w rozdziałach poświęconych części programowej pracy.
