\documentclass[a4paper,12pt]{article}

% Polski język i kodowanie
\usepackage[utf8]{inputenc}
\usepackage[T1]{fontenc}
\usepackage[polish]{babel}

% Podstawowe pakiety
\usepackage{amsmath, amssymb}
\usepackage{graphicx}
\usepackage{hyperref}
\usepackage{lmodern}
\usepackage{csquotes}
 \usepackage{siunitx}
 \sisetup{per-mode=symbol,detect-all}
\usepackage{microtype}
\usepackage{graphicx}
\usepackage{float}
\usepackage{booktabs}
\usepackage{hyperref}
\usepackage{geometry}
\geometry{margin=2.5cm}
\usepackage[backend=bibtex,style=numeric,sortcites,sorting=nty,backref,natbib,hyperref]{biblatex}

\addbibresource{bibliografia.bib}




\title{Projekt stanowiska pomiarowego podstawowych wielkości fizycznych wykorzystywanych w systemach HVAC
\\\large w oparciu o zestaw STM32F746G-Discovery z wyświetlaczem TFT}
\author{Imię Nazwisko}
\date{\today}

\begin{document}
\maketitle

\begin{abstract}
Praca przedstawia projekt oraz weryfikację stanowiska pomiarowego dla wybranych wielkości fizycznych typowych w systemach HVAC (temperatura, wilgotność, ciśnienie, przepływ oraz sygnały sterujące \SI{0}{\volt}–\SI{10}{\volt}). Rozwiązanie bazuje na zestawie uruchomieniowym STM32F746G-Discovery z wyświetlaczem TFT oraz na dedykowanej karcie pomiarowo–wyjściowej \SI{0}{\volt}–\SI{10}{\volt} zaprojektowanej w KiCad. Oprogramowanie wykonano w środowisku Zephyr RTOS z wykorzystaniem sterowników peryferiów i biblioteki LVGL do obsługi interfejsu graficznego.
Przedstawiono wymagania funkcjonalne, projekt części analogowej (tory wejściowe z ochroną i skalowaniem, tor wyjściowy \SI{0}{\volt}–\SI{10}{\volt}), architekturę oprogramowania (wątki RTOS, kolejki, sterowniki), a także wyniki walidacji dokładności i powtarzalności pomiarów na podstawie wzorców i porównania z przyrządami referencyjnymi. 
\end{abstract}

\tableofcontents

\section{Wstęp}
Celem pracy jest zaprojektowanie i weryfikacja stanowiska pomiarowego dla podstawowych wielkości fizycznych spotykanych w HVAC oraz interfejsu sterującego \SI{0}{\volt}–\SI{10}{\volt}, z wykorzystaniem zestawu STM32F746G-Discovery (STM32F746NG, Cortex-M7) i dedykowanego modułu pomiarowego PCB. Motywacją jest potrzeba ekonomicznego, dydaktycznego stanowiska do testów i demonstracji algorytmów sterowania.

Omówiono kontekst przemysłowy sygnałów \SI{0}{\volt}–\SI{10}{\volt}, przegląd czujników oraz wymagania co do dokładności i izolacji torów.
\section{Stan wiedzy i rozwiązania pokrewne}
Krótki przegląd: interfejsy analogowe w HVAC (0–10 V, 4–20 mA), standardy i zalecenia, dostępne moduły komercyjne oraz przykłady platform STM32 z wyświetlaczem TFT.
\section{Wymagania}
\subsection{Funkcjonalne}
\begin{itemize}
  \item Pomiar napięć wejściowych w zakresie \SIrange{0}{10}{\volt} z rozdzielczością $\leq$ \SI{10}{\milli\volt}.
  \item Generacja sygnału wyjściowego \SIrange{0}{10}{\volt} obciążalność $\geq$ \SI{5}{\milli\ampere}.
  \item GUI na TFT: wizualizacja trendów, konfiguracja kanałów.
\end{itemize}
\subsection{Niefunkcjonalne}
EMC, bezpieczeństwo, ESD, kalibracja (offset/gain), testowalność.
\section{Projekt części sprzętowej}
\subsection{Zestaw bazowy}
Użyto zestawu STM32F746G-Discovery z wyświetlaczem TFT
\subsection{Tor pomiarowy 0–10 V}
Opis idei: dzielnik wejściowy, ochrona ESD, filtr RC antyaliasing, bufor operacyjny o niskim prądzie biasu,
dopasowanie do ADC MCU lub zewnętrznego ADC (jeśli w schemacie). Wstaw tutaj referencje do konkretnych wzmacniaczy/oporników z BOM.

\subsection{Tor wyjściowy 0–10 V}
Opis: DAC (wewnętrzny z wzmacniaczem/zewnętrzny), wzmacniacz operacyjny w konfiguracji nieodwracającej,
zasilanie symetryczne/podwyższające jeśli wymagane, ograniczenie prądowe, filtr wyjściowy. Podmień TODO na rzeczywiste MPN.

\subsection{Zasilanie i separacja}
Regulatory LDO/DC-DC, separacja AGND/DGND, gwiazda masy, ekranowanie przewodów pomiarowych.

\subsection{PCB i testy}
Reguły DRC, szerokości ścieżek, pętle masy. Zrzuty z KiCada (\texttt{figures/}).
\section{Oprogramowanie (Zephyr RTOS)}
\subsection{Architektura}
Wątki: akwizycja pomiarów, filtracja, GUI (LVGL), sterowanie wyjściem 0–10 V. IPC: kolejki, semafory.

\subsection{Sterowniki}
Konfiguracja ADC/DAC/I2C/SPI w Zephyr (DeviceTree), obsługa dotyku i TFT, logowanie.

\subsection{GUI}
Ekrany: podgląd kanałów, konfiguracja zakresów, kalibracja dwu-punktowa.

\subsection{Testy}
Testy jednostkowe (zassert), pomiary jitteru i latencji, profilowanie czasu CPU.
\section{Walidacja}
\subsection{Metodyka}
Źródła wzorcowe (kalibrator napięcia), obciążenie dla wyjścia 0–10 V, środowisko testowe.

\subsection{Wyniki}
Tabele dokładności, histogram odchyleń, niepewność typu A/B, budżet niepewności.

\subsection{Dyskusja}
Ograniczenia, dryft temperaturowy, histereza, propozycje ulepszeń.

\cite{Analog_0_10V_Guideline}



\printbibliography
\end{document}