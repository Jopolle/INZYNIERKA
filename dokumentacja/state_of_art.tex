\chapter{Stan wiedzy i założenia projektowe}

Celem projektu jest przede wszystkim zaprojektowanie i uruchomienie stanowiska
laboratoryjnego, a nie opracowanie nowego modelu teoretycznego. Z tego powodu
przegląd literatury ma charakter praktyczny i koncentruje się na zagadnieniach,
które były wykorzystane w projekcie: sygnał w standardzie 0-10~V używany w automatyce przemysłowej, podstawowych algorytmach regulacji oraz ogólnych zasadach projektowania
płytek PCB dla układów mieszanych analogowo--cyfrowych. 

\section{Sygnał 0--10~V i sterowanie HVAC}

W automatyce budynkowej i innych układach automatyki jednym z najczęściej spotykanych sygnałów sterujących jest
sygnał napięciowy 0--10~V lub sygnał prądowy 4-20~mA. W materiałach producentów czujników, przetworników oraz w
opracowaniach dotyczących automatyki budynkowej podkreśla się, że sygnał 0--10~V
jest powszechnie używany do sterowania siłownikami przepustnic i zaworów, falownikami oraz różnymi czujnikami stosowanymi w instalacjach
HVAC\cite{veris_010v,jakar_010v,sevensesnor_010v}. Główna zaleta takiego sygnału
to prostota: praktycznie każdy sterownik PLC lub system BMS potrafi wygenerować
albo zmierzyć napięcie 0--10~V, a jego interpretacja jest intuicyjna ( 0--100~\%
odpowiada 0--10~V).

W literaturze naukowej dotyczącej sterowania HVAC większość uwagi poświęca się
modelom cieplnym i algorytmom regulacji, natomiast warstwa sprzętowa jest zwykle opisywana
dość ogólnie. W artykułach dotyczących energooszczędnego sterowania HVAC stosuje
się klasyczne sygnały analogowe (napięciowe lub prądowe)\cite{kirubakaran2015,almabrok2018}.
W niniejszej pracy ten tor został celowo ,,wyciągnięty na wierzch'' w postaci
osobnej płytki dydaktycznej, tak aby można było go obserwować i modyfikować podczas
zajęć laboratoryjnych.

\section{Regulacja PI/PID i sekwencje pracy central HVAC}

W publikacjach dotyczących sterowania HVAC najczęściej stosuje się regulatory PI
lub PID. Do ich strojenia wykorzystuje się zarówno proste metody (np.\ Ziegler--Nichols),
ale również bardziej zaawansowane algorytmy optymalizacyjne\cite{almabrok2018,qu2004,fernandes2019}.
Przykładowo Almabrok i in.\ prezentują szybką metodę strojenia regulatora PID
dla systemu HVAC z użyciem algorytmu optymalizacyjnego\cite{almabrok2018}. Zminy termodynamiczne w budynkach są na ogół procesem powolnym, dlatego algorytmy przyspieszające optymalne strojenie regulatorów są głównym motywem przewodnim prac badawczo--naukowych w tej dziedzinie.

W przypadku logiki działania całych central wentylacyjnych, często odwołuje się
do ustandaryzowanych sekwencji pracy opisanych w wytycznych ASHRAE, w szczególności
w dokumencie Guideline~36\cite{ashrae_g36}. Wytyczne te definiują gotowe sekwencje
sterowania nagrzewnicą, chłodnicą, odzyskiem ciepła i bypassem, uwzględniając pasmo
martwe oraz warunki bezpieczeństwa. W projekcie zastosowano uproszczony wariant
takiej sekwencji: pojedynczy regulator PI temperatury, którego wyjście jest
podzielone na przedziały odpowiadające grzaniu, chłodzeniu, odzyskowi ciepła oraz
pasmowi martwemu.

\section{Sterowniki HVAC na mikrokontrolerach i RTOS}

W literaturze można znaleźć przykłady implementacji sterowania HVAC z użyciem
mikrokontrolerów oraz prostych systemów operacyjnych czasu rzeczywistego. W pracy
Fernandesa opisano regulator PID przepływu powietrza w instalacji wentylacyjnej,
zaimplementowany na mikrokontrolerze, z lokalnym pomiarem oraz sterowaniem pracą
wentylatora\cite{fernandes2019}. Układ ten jest zbliżony koncepcyjnie do niniejszego projektu, obejmuje jedną pętlę regulacji, czujnik, element wykonawczy i prosty interfejs użytkownika.

W projekcie jako środowisko firmware’u wykorzystano Zephyr RTOS, lekki
system operacyjny czasu rzeczywistego przeznaczony do układów wbudowanych, obsługujący
wiele architektur i posiadający rozbudowany zestaw sterowników\cite{zephyr_intro,zephyr_docs}.
Dokumentacja Zephyra pokazuje typowe podejście do struktury aplikacji: logika jest
podzielona na wątki, natomiast sprzęt (np.\ magistrale SPI, wyświetlacze) opisuje się w
drzewie urządzeń (Devicetree). Do realizacji interfejsu HMI wykorzystano bibliotekę
LVGL, czyli popularną otwartą bibliotekę graficzną dla mikrokontrolerów\cite{lvgl_intro,lvgl_zephyr}.
LVGL udostępnia gotowe widżety (przyciski, wykresy, listy), dzięki czemu można skupić
się na logice HVAC, zamiast implementować od podstaw warstwę graficzną.

\section{Ramy projektowania PCB dla układów mieszanych}

Istotnym elementem projektu jest płytka PCB zawierająca zarówno sygnały analogowe 0--10~V,
jak i cyfrowe interfejsy SPI. W literaturze dotyczącej projektowania układów
mieszanych powtarza się kilka podstawowych zaleceń: logiczny podział
płytki na część analogową i cyfrową, kontrola powrotu prądów w masie oraz rozsądne
prowadzenie zasilania\cite{ott_mixed,adi_an404,microchip_analog_guide}. 
Ott \cite{ott_mixed} zwraca uwagę, że zamiast dzielić masę na dwie osobne płaszczyzny, korzystniej jest utrzymać jedną
wspólną masę i wydzielić część analogową oraz cyfrową głównie geometrycznie oraz
odpowiednim prowadzeniem ścieżek.

W notach aplikacyjnych firm Analog Devices i Microchip przedstawiono przykładowe
projekty płytek dla układów mieszanych, gdzie pokazano m.in.\ sposób umieszczania
przetworników ADC na granicy stref analog/digital, prowadzenia linii SPI oraz
stosowania przelotek łączących pola masy\cite{adi_an404,microchip_analog_guide}. W niniejszym
projekcie przyjęto podobne podejście: sygnały 0--10~V znajdują się w wydzielonych częściach
płytki z osobnym polem masy analogowej GNDA, interfejsy SPI umieszczono bliżej
złącza do płytki STM32F746G-DISCO, natomiast przetworniki cyfrowo/analogowe i analogowo/cyfrowe pełnią rolę ,,mostu'' pomiędzy
częścią analogową a cyfrową\cite{TI_ADS8688,TI_DAC7568}.
